\chapter{Introduction}

\section{Context}
Small businesses such as barber shops can oftentimes be faced with less than ideal situations if their operation is walk-ins only. Things like lack of business insights, customer friction due to inability to be serviced when they wish, manual bookkeeping, etc. In today's day and age, digitalizing your business is a must, not only because it's what everyone is already doing, but also because it builds a bridge between the business and its customers, leading to a more established presence in the market.

By leveraging technologies such as \textit{TRiM}, we are able to scale the business and bring real value to the table, and not just financial value, but also operational value as this tool aims to improve every corner the business. But what exactly is \textit{TRiM}? This platform aims to support all operations of service-based businesses such as barber shops, from customer bookings, to payment integration, and employee management. The goal is to allow multiple businesses to use the platform independently so that each one of them can have access to all the features offered by \textit{TRiM}.

\section{Project Objectives}
The objective of this project is clear: to deliver a real and robust software solution to local service-based businesses. These are the goals:
\begin{itemize}
    \item Allow customers to book a service (logged in or as a guest) at a time that suits them with a professional of their choosing while preventing common issues such as double bookings.
    \item Support payment processing.
    \item Allow business owners to oversee operations by adding services/employees and prices.
\end{itemize}

\section{Dissertation Focus}
The focus of this dissertation is mostly architectural, with a focus on database design patterns and data isolation techniques. With \textit{TRiM}, we're exploring how to create a multi-tenant software application that's scalable and robust across the board, but also leverages efficient design patterns that allows us to grow without massive costs or performance overhead. We also investigate how to make sure multiple business can have access to this system whilst maintaining performance and security at the top of our priorities.

\section{Success Metrics}
Describe sucess metrics.

\section{Chapters Overview}
Give an overview of all the chapters.
